\input{mmd-myarticle-header}
\def\mytitle{My Title}
\def\myauthor{Pedro Jordano      }
\def\latexaffiliation{Integrative Ecology Group, Estación Biológica de Doñana, Consejo Superior de Investigaciones Científicas (EBD-CSIC), Avenida Americo Vespucio s\slash n, E-41092 Sevilla, Spain}
\def\mydate{August 28, 2015}
\def\latexmode{memoir}
\def\email{jordano@ebd.csic.es}
\def\format{complete}
\def\latexxslt{xhtml2latex.xslt}
\def\bibliocommand{\bibliography{MS}}
\def\bibliographytitle{Bibliography}
\def\bibliostyle{bes}
\input{mmd-myarticle-begin-doc}

%\chapter{Sampling networks of ecological interactions}
%\label{samplingnetworksofecologicalinteractions}

\emph{Correspondence}: Pedro Jordano, Fax: + 34 954 62 11 25.\\
E-mail: \texttt{jordano@ebd.csic.es} 

\emph{Running headline}: Headline title 

\emph{Keywords}: complex networks, mutualism, plant-animal interactions, frugivory, pollination, seed dispersal, food webs 

\begin{center}\rule{3in}{0.4pt}\end{center}


\section{Summary}
\label{summary}

\begin{enumerate}
\item .

\item .

\item .

\item .

\item .

\end{enumerate}

\begin{center}\rule{3in}{0.4pt}\end{center}


\section{Introduction}
\label{introduction}

\begin{adjustwidth}{2.5em}{2.5em}
\begin{verbatim}

MYTEXT.   
A & B (2011).   

\end{verbatim}
\end{adjustwidth}

Biodiversity assessment aims at sampling individuals in collections and determining the number of species represented. Given that, by definition, samples are incomplete, these collections enumerate a lower number of the species actually present. The ecological literature dealing with robust estimators of species richness and diversity in collections of individuals is immense, and a number of useful approaches have been used to obtain such estimates ~\citep{Magurran:1988mm,Gotelli:2001uo,Hortal:2006dc,Colwell:2009gv,Gotelli:2011tb}. Recent effort has been also focused at defining essential biodiversity variables (EBV) ~\citep{Pereira:2013ji} that can be sampled and measured repeatedly to complement biodiversity estimates. Yet sampling species or taxa-specific EBVs is just probing a single component of biodiversity; interactions among species are another fundamental component, the one that supports the existence of species. For example, the extinction of interactions represents a dramatic loss of biodiversity because it entails the loss of fundamental ecological functions ~\citep{ValienteBanuet:2014bw}. This missed component of biodiversity loss, the extinction of ecological interactions, very often accompanies, or even precedes, species disappearance. Interactions among species are a key component of biodiversity and here I aim to show that most problems associated to sampling interactions in natural communities have to do with problems associated to sampling species diversity. I consider pairwise interactions among species at the habitat level, in the context of alpha diversity and the estimation of local interaction richness from sampling data ~\citep{Mao:2005tka}. In the first part I provide a succinct overview of previous work addressing sampling issues for ecological interaction networks. In the second part I discuss specific rationales for sampling the biodiversity of ecological interactions. 

\begin{center}\rule{3in}{0.4pt}\end{center}


\section{Material and Methods}
\label{materialandmethods}

AAA

\begin{center}\rule{3in}{0.4pt}\end{center}


\section{Results}
\label{results}

AAA

\begin{center}\rule{3in}{0.4pt}\end{center}


\section{Discussion}
\label{discussion}

AAA 

\begin{center}\rule{3in}{0.4pt}\end{center}


\section{Acknowledgements}
\label{acknowledgements}

AAA 

\begin{center}\rule{3in}{0.4pt}\end{center}


\section{Data archiving}
\label{dataarchiving}

\begin{center}\rule{3in}{0.4pt}\end{center}


\section{Tables}
\label{tables}

Table 1. 

\begin{table}[htbp]
\begin{minipage}{\linewidth}
\setlength{\tymax}{0.5\linewidth}
\centering
\small
\caption{Table 1. Simple\_table.}
\label{table1.simple_table.}
\begin{tabulary}{\textwidth}{@{}LCR@{}} \toprule
First Header&Second Header&Third Header\\
\midrule
First row&Data&Very long data entry\\
Second row&\textbf{Cell}&\emph{Cell}\\

\bottomrule

\end{tabulary}
\end{minipage}
\end{table}

\begin{center}\rule{3in}{0.4pt}\end{center}


Table 2. 

\begin{table}[htbp]
\begin{minipage}{\linewidth}
\setlength{\tymax}{0.5\linewidth}
\centering
\small
\caption{Table 2. Prototype table}
\label{reference_table}
\begin{tabulary}{\textwidth}{@{}LCR@{}} \toprule
&\multicolumn{2}{c}{Grouping}\\
First Header&Second Header&Third Header\\
\midrule
Content&\multicolumn{2}{c}{\emph{Long Cell}}\\
Content&\textbf{Cell}&Cell\\
New section&More&Data\\

\bottomrule

\end{tabulary}
\end{minipage}
\end{table}

\begin{center}\rule{3in}{0.4pt}\end{center}


\section{Figures}
\label{figures}

Figure 1. 

Figure 2. 

Figure 3. 

\begin{center}\rule{3in}{0.4pt}\end{center}


\section{Supplementary Material}
\label{supplementarymaterial}

\begin{center}\rule{3in}{0.4pt}\end{center}


\input{mmd-memoir-footer}

\end{document}